\documentclass[11pt,letterpaper]{article}
\usepackage[lmargin=1in,rmargin=1in,tmargin=1in,bmargin=1in]{geometry}
%For systems of equations
\usepackage{systeme}
%For figures
\usepackage{graphicx}
\usepackage{subcaption}
% -------------------
% Packages
% -------------------
\usepackage{
	amsmath,			% Math Environments
	amssymb,			% Extended Symbols
	enumerate,		    % Enumerate Environments
	graphicx,			% Include Images
	lastpage,			% Reference Lastpage
	multicol,			% Use Multi-columns
	multirow			% Use Multi-rows
}


% -------------------
% Font
% -------------------
\usepackage[T1]{fontenc}
\usepackage{charter}


% -------------------
% Commands
% -------------------
\newcommand{\quiz}[2]{\noindent\textbf{Name: }\makebox[8cm]{\hrulefill} \hfill \textbf{Algebra II} \\  \textbf{Date: } \hfill \textbf{Quiz #2}\\}

\newcommand{\prob}{\noindent\textbf{Problem. }}
\newcounter{problem}
\newcommand{\problem}{
	\stepcounter{problem}%
	\noindent \textbf{ Problem \theproblem. }        
}
\newcommand{\pspace}{\par\vspace{\baselineskip}}
\newcommand{\ds}{\displaystyle}


% -------------------
% Header & Footer
% -------------------
\usepackage{fancyhdr}

\fancypagestyle{pages}{
	%Headers
	\fancyhead[L]{}
	\fancyhead[C]{}
	\fancyhead[R]{}
\renewcommand{\headrulewidth}{0pt}
	%Footers
	\fancyfoot[L]{}
	\fancyfoot[C]{}
	\fancyfoot[R]{}
\renewcommand{\footrulewidth}{0.0pt}
}
\headheight=0pt
\footskip=14pt

\pagestyle{pages}


% -------------------
% Content
% -------------------
\begin{document}
\quiz{\#}{2}


% Question
%\prob This is an unnumbered problem. \pspace


% Question 1
\problem Write $(x - 1)^2 (x + 1)^2$ in standard form. That is, as a polynomial in descending powers of $x$. (Your answer should have the form $a_nx^n + a_{n-1}x^{n-1} + \dots + a_2 x^2 + a_1 x + a_0$. \textit{Hint}: There's an easy way to do this, and a hard way.)

$$(x - 1)^2 (x + 1)^2$$
$$= ((x-1)(x+1))^2$$
$$= (x^2 - 1)^2$$
$$= (x^2 - 1)(x^2-1)$$
$$ = x^4 - 2x^2 + 1$$

\vspace{.5cm}


% Question 2
\problem Factor $25x^2 - 36y^2$ completely.
$$25x^2 - 36y^2$$
$$= (5x)^2 - (6y)^2$$
$$= (5x + 6y)(5x - 6y)$$

\vspace{.5cm}


% Question 3
\problem Factor $16z^4 - 81w^4$ completely.
$$16z^4 - 81w^4$$
$$= (4z^2)^2 - (9w^2)^2$$
$$= (4z^2 + 9w^2)(4z^2 - 9w^2)$$
$$= (4z^2 + 9w^2)((2z)^2 - (3w)^2)$$
$$= (4z^2 + 9w^2)(2z+3w)(2z - 3w)$$

\vspace{.5cm}


% Question 4
\problem Factor $y^3 + y^2 - 4y - 4$ completely.
$$y^3 + y^2 - 4y - 4$$
$$= y^3 + y^2 - 4(y + 1)$$
$$= y^2(y + 1) - 4(y + 1)$$
$$= (y^2 - 4)(y + 1)$$
$$= (y + 2)(y - 2)(y + 1)$$

\vspace{.5cm}

\newpage

% Question 5
\problem Factor $10x^2 + 11x - 6$ completely.
$$10x^2 + 11x - 6$$
$$= (5x - 2)(2x + 3)$$

\vspace{.5cm}


% Question 6
\problem Factor $5x^3 + 40y^3$ completely.
$$5x^3 + 40y^3$$
$$= 5(x^3 + 8y^3)$$
$$= 5(x^3 + (2y)^3)$$
$$= 5(x + 2y)(x^2 - (x)(2y) + (2y)^2)$$
$$= 5(x + 2y)(x^2 - 2xy + 4y^2)$$

\vspace{.5cm}


% Question 7
\problem Let $f(x) = \frac{1}{x^2 - 2x - 15}$. Write the domain of $f$ in \textit{interval notation}.
\\
\\
We need to find the allowable inputs to $f$. Clearly the only problem we'll have is if the denominator is equal to $0$. We'll be able to take all real numbers as inputs except for the solutions to the equation
$$x^2 - 2x - 15 = 0.$$
And since
$$x^2 - 2x - 15 = (x+3)(x-5),$$
we have 
$$(x+3)(x-5)=0$$
precisely when $x = -3$ or $x = 5$. Therefore all real numbers except $-3$ and $5$ are allowable inputs to $f$. To put this in interval notation, we think of the real numbers as an interval, $(-\infty, \infty)$, and pull out the numbers $-3$ and $5$, leaving us 
$$(-\infty, -3) \cup (-3, 5) \cup (5, \infty).$$
Thus,
$$\text{Dom}(f) = (-\infty, -3) \cup (-3, 5) \cup (5, \infty)$$

\vspace{2.5cm}





\end{document}