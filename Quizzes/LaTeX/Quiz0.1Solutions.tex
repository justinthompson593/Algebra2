\documentclass[11pt,letterpaper]{article}
\usepackage[lmargin=1in,rmargin=1in,tmargin=1in,bmargin=1in]{geometry}
%For systems of equations
\usepackage{systeme}
%For figures
\usepackage{graphicx}
\usepackage{subcaption}
% -------------------
% Packages
% -------------------
\usepackage{
	amsmath,			% Math Environments
	amssymb,			% Extended Symbols
	enumerate,		    % Enumerate Environments
	graphicx,			% Include Images
	lastpage,			% Reference Lastpage
	multicol,			% Use Multi-columns
	multirow			% Use Multi-rows
}


% -------------------
% Font
% -------------------
\usepackage[T1]{fontenc}
\usepackage{charter}


% -------------------
% Commands
% -------------------
\newcommand{\quiz}[2]{\noindent\textbf{Name: }\makebox[8cm]{\hrulefill} \hfill \textbf{Algebra II} \\  \textbf{Date: } \hfill \textbf{Quiz #2}\\}

\newcommand{\prob}{\noindent\textbf{Problem. }}
\newcounter{problem}
\newcommand{\problem}{
	\stepcounter{problem}%
	\noindent \textbf{Problem \theproblem. }%
}
\newcommand{\pspace}{\par\vspace{\baselineskip}}
\newcommand{\ds}{\displaystyle}


% -------------------
% Header & Footer
% -------------------
\usepackage{fancyhdr}

\fancypagestyle{pages}{
	%Headers
	\fancyhead[L]{}
	\fancyhead[C]{}
	\fancyhead[R]{}
\renewcommand{\headrulewidth}{0pt}
	%Footers
	\fancyfoot[L]{}
	\fancyfoot[C]{}
	\fancyfoot[R]{}
\renewcommand{\footrulewidth}{0.0pt}
}
\headheight=0pt
\footskip=14pt

\pagestyle{pages}


% -------------------
% Content
% -------------------
\begin{document}
\quiz{\#}{0.1}


% Question
%\prob This is an unnumbered problem. \pspace


% Question 1
\problem The following sets of points are in the form $(x,y)$ where the $x$ values are considered as inputs and $y$ values are outputs and thereby define a rule. Determine which of these define functions and for those that don't, explain why.\\
	\begin{enumerate}[(a)]
	\item $\left\{ (1,2), (2,3), (3,4), (4,5)\right\}$ defines a function
	\item $\left\{ (4,2), (3,3), (3,2), (1,5)\right\}$ is not a function ($3$ maps to $2$ different outputs)
	\item $\left\{ (2.01,1.32), (2.02,3.1), (2.03,2.4), (2.001,5.7)\right\}$ defines a function
	\item $\left\{ (2.01,2.02), (2.02,2.02), (2.03,2.02), (2.001,2.02)\right\}$ defines a function
	\item $\left\{ (\text{red},\beta), (\text{green},\gamma), (\text{blue},\alpha), (\text{white},\delta)\right\}$ defines a function
	\item $\left\{ (2^3,2), (3,3), (2,3), (8,5)\right\}$ is not a function ($2^3 = 8$ maps to $2$ different numbers
	\item $\left\{ (2^3,2), (3,3), (2,3), (8,2)\right\}$ defines a function ($2^3 = 8$ maps to only one number)
	\end{enumerate} 

\vspace{.5cm}



% Question 2
\problem The following graphs define relationships between inputs (on the horizontal, or x, axis) and outputs (on the y axis). Determine which ones define functions and which do not. Briefly explain your choices. \\
\begin{enumerate}[(a)]
	\item defines a function
	\item defines a function
	\item is not a function (fails vertical line test / has 2 outputs for many inputs)
	\item is not a function (fails vertical line test / has multiple outputs for many inputs)
	\item defines a function
	\item is not a function (fails vertical line test / has multiple outputs for all inputs $\ge 0$)
	\end{enumerate} 

\newpage
%\vspace{.5cm}


% Question 3
\problem Graph the following lines and write its slope. Be sure to label your axes, and if applicable, the $x$ and $y$ intercepts. \\
\begin{enumerate}[(a)]
	\item $y = -2x$: $x$-intercept $= 0$, $y$-intercept $= 0$, slope = $-2$, see video for graph
	\item $y = 5$: $x$-intercept does not exist, $y$-intercept $= 5$, slope = $0$, see video for graph
	\item $y = 2x + 3$: $x$-intercept $= \frac{-3}{2}$, $y$-intercept $= 3$, slope = $2$, see video for graph
	\item $x + y = 1$: $x$-intercept $= 1$, $y$-intercept $= 1$, slope = $-1$, see video for graph
	\item $x = -3$: $x$-intercept $= -3$, $y$-intercept does not exist, slope is undefined, see video for graph
	\item $2x - 3y = 5$: $x$-intercept $= \frac{5}{2}$, $y$-intercept $= \frac{-5}{3}$, slope = $\frac{2}{3}$, see video for graph
	\item Do these lines define functions? How do you know? For each line, what's the domain? What's the range?\\
	They all define functions except for the vertical line $x = -3$ in part (e). The domain of each one is all real numbers, $( -\infty, \infty)$, except for the vertical line $x = -3$ in part (e) whose domain is the singleton $\{ 3\}$. The range of each one is all real numbers, $( -\infty, \infty)$, except for the horizontal line $y = 5$ in part (b), whose range is the singleton $\{5\}$.
\end{enumerate} 

\vspace{.5cm}


%Q4
\problem Graph the following functions. Be sure to label your axes, and if applicable, the axes intercepts. \\
\begin{enumerate}[(a)]
	\item $f(x) = x^2$: $x$-intercept $= 0$, $f$-intercept $= 0$, see video for graph 
	\item $g(z) = -z^2$: $z$-intercept $= 0$, $g$-intercept $= 0$, see video for graph
	\item $h(a) = a^2 - 4$: $a$-intercepts $= \{-2,2\}$, $h$-intercept $= -4$, see video for graph
	\item $y(x) = \frac{1}{2}x - 1$: $x$-intercept $= 2$, $y$-intercept $= -1$, see video for graph
	\item $z(t) = 1 - t^2$: $t$-intercepts $= \{-1, 1\}$, $z$-intercept $= 1$, see video for graph
	\item $q(p) = -\frac{3}{2}p + 1$: $p$-intercept $= \frac{2}{3}$, $q$-intercept $= 1$, see video for graph
\end{enumerate}

\vspace{.5cm}

%Q5
\problem Let $f(x) = x + 1$, $g(x) = \sqrt{x}$, and $h(x) = \frac{1}{x}$. Compute the following.\\
\begin{enumerate}[(a)]
	\item $f(-2) = -1$
	\item $g(4) = 2$
	\item $h(0)$ is not defined
	\item $f(g(x)) = \sqrt{x} + 1$, Domain: $[0, \infty)$
	\item $g(f(x)) = \sqrt{x+1}$, Domain: $[-1, \infty)$
	\item $h(g(f(x))) = \frac{1}{\sqrt{x+1}}$, Domain: $(-1, \infty)$
	\item $f(h(g(x))) = \frac{1}{\sqrt{x}} + 1$, Domain: $(0, \infty)$
	\item Domains are indicated above
	\item $f^{-1}(x) = x - 1$
	\item $g^{-1}(x) = x^2$
	\item $h^{-1}(x) = \frac{1}{x}$
\end{enumerate}



























\end{document}