\documentclass[11pt,letterpaper]{article}
\usepackage[lmargin=1in,rmargin=1in,tmargin=1in,bmargin=1in]{geometry}
%For systems of equations
\usepackage{systeme}
%For figures
\usepackage{graphicx}
\usepackage{subcaption}
% -------------------
% Packages
% -------------------
\usepackage{
	amsmath,			% Math Environments
	amssymb,			% Extended Symbols
	enumerate,		    % Enumerate Environments
	graphicx,			% Include Images
	lastpage,			% Reference Lastpage
	multicol,			% Use Multi-columns
	multirow			% Use Multi-rows
}


% -------------------
% Font
% -------------------
\usepackage[T1]{fontenc}
\usepackage{charter}


% -------------------
% Commands
% -------------------
\newcommand{\quiz}[2]{\noindent\textbf{Name: }\makebox[8cm]{\hrulefill} \hfill \textbf{Algebra II} \\  \textbf{Date: } \hfill \textbf{Quiz #2}\\}

\newcommand{\prob}{\noindent\textbf{Problem. }}
\newcounter{problem}
\newcommand{\problem}{
	\stepcounter{problem}%
	\noindent \textbf{Problem \theproblem. }%
}
\newcommand{\pspace}{\par\vspace{\baselineskip}}
\newcommand{\ds}{\displaystyle}


% -------------------
% Header & Footer
% -------------------
\usepackage{fancyhdr}

\fancypagestyle{pages}{
	%Headers
	\fancyhead[L]{}
	\fancyhead[C]{}
	\fancyhead[R]{}
\renewcommand{\headrulewidth}{0pt}
	%Footers
	\fancyfoot[L]{}
	\fancyfoot[C]{}
	\fancyfoot[R]{}
\renewcommand{\footrulewidth}{0.0pt}
}
\headheight=0pt
\footskip=14pt

\pagestyle{pages}


% -------------------
% Content
% -------------------
\begin{document}
\quiz{\#}{0.1}


% Question
%\prob This is an unnumbered problem. \pspace


% Question 1
\problem The following sets of points are in the form $(x,y)$ where the $x$ values are considered as inputs and $y$ values are outputs and thereby define a rule. Determine which of these define functions and for those that don't, explain why.\\
	\begin{enumerate}[(a)]
	\item $\left\{ (1,2), (2,3), (3,4), (4,5)\right\}$
	\item $\left\{ (4,2), (3,3), (3,2), (1,5)\right\}$
	\item $\left\{ (2.01,1.32), (2.02,3.1), (2.03,2.4), (2.001,5.7)\right\}$
	\item $\left\{ (2.01,2.02), (2.02,2.02), (2.03,2.02), (2.001,2.02)\right\}$
	\item $\left\{ (\text{red},\beta), (\text{green},\gamma), (\text{blue},\alpha), (\text{white},\delta)\right\}$
	\item $\left\{ (2^3,2), (3,3), (2,3), (8,5)\right\}$
	\item $\left\{ (2^3,2), (3,3), (2,3), (8,2)\right\}$
	\end{enumerate} 

\vspace{.5cm}



% Question 2
\problem The following graphs define relationships between inputs (on the horizontal, or x, axis) and outputs (on the y axis). Determine which ones define functions and which do not. Briefly explain your choices. \\
\begin{figure}[h!]
  \centering
  \begin{subfigure}[b]{0.3\linewidth}
    \includegraphics[width=\linewidth]{func1.png}
     \caption{}
  \end{subfigure}
  \begin{subfigure}[b]{0.3\linewidth}
    \includegraphics[width=\linewidth]{func2.png}
    \caption{}
  \end{subfigure}
  \begin{subfigure}[b]{0.3\linewidth}
    \includegraphics[width=\linewidth]{func3.png}
    \caption{}
  \end{subfigure}
  \begin{subfigure}[b]{0.3\linewidth}
    \includegraphics[width=\linewidth]{func4.png}
     \caption{}
  \end{subfigure}
  \begin{subfigure}[b]{0.3\linewidth}
    \includegraphics[width=\linewidth]{func5.png}
    \caption{}
  \end{subfigure}
  \begin{subfigure}[b]{0.3\linewidth}
    \includegraphics[width=\linewidth]{func6.png}
    \caption{}
  \end{subfigure}
\end{figure}

\newpage
%\vspace{.5cm}


% Question 3
\problem Graph the following lines and write its slope. Be sure to label your axes, and if applicable, the $x$ and $y$ intercepts. \\
\begin{enumerate}[(a)]
	\item $y = -2x$
	\item $y = 5$
	\item $y = 2x + 3$
	\item $x + y = 1$
	\item $x = -3$
	\item $2x - 3y = 5$
	\item Do these lines define functions? How do you know? For each line, what's the domain? What's the range?
\end{enumerate} 

\vspace{.5cm}


%Q4
\problem Graph the following functions. Be sure to label your axes, and if applicable, the axes intercepts. \\
\begin{enumerate}[(a)]
	\item $f(x) = x^2$
	\item $g(z) = -z^2$
	\item $h(a) = a^2 - 4$
	\item $y(x) = \frac{1}{2}x - 1$
	\item $z(t) = 1 - t^2$ (Hint: consider $g(t) = -t^2$ from above. Then $g(t) + 1 = -t^2 + 1 = 1 - t^2$ is just a shift of $g$ up by 1.)
	\item $q(p) = -\frac{3}{2}p + 1$
\end{enumerate}

\vspace{.5cm}

%Q5
\problem Let $f(x) = x + 1$, $g(x) = \sqrt{x}$, and $h(x) = \frac{1}{x}$. Compute the following.\\
\begin{enumerate}[(a)]
	\item $f(-2)$
	\item $g(4)$
	\item $h(0)$
	\item $f(g(x))$
	\item $g(f(x))$
	\item $h(g(f(x)))$
	\item $f(h(g(x)))$
	\item For (d) - (g) state the domain of each function in interval notation. (Example: the domain of $h(f(x)) = \frac{1}{x+1}$ would be all real numbers such that the denominator, $x+1$, is not $0$. So we must have $x+1 \not = 0$, therefore $x \not = -1$. Then all real numbers except for $-1$ are allowable inputs to $h(f(x))$. In interval notation, we have $(-\infty, -1) \cup (-1, \infty)$.)
	\item $f^{-1}(x)$
	\item $g^{-1}(x)$
	\item $h^{-1}(x)$
\end{enumerate}



























\end{document}