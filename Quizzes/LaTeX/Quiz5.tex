\documentclass[11pt,letterpaper]{article}
\usepackage[lmargin=1in,rmargin=1in,tmargin=1in,bmargin=1in]{geometry}
%For systems of equations
\usepackage{systeme}
%For figures
\usepackage{graphicx}
\usepackage{subcaption}
% -------------------
% Packages
% -------------------
\usepackage{
	amsmath,			% Math Environments
	amssymb,			% Extended Symbols
	enumerate,		    % Enumerate Environments
	graphicx,			% Include Images
	lastpage,			% Reference Lastpage
	multicol,			% Use Multi-columns
	multirow			% Use Multi-rows
}


% -------------------
% Font
% -------------------
\usepackage[T1]{fontenc}
\usepackage{charter}


% -------------------
% Commands
% -------------------
\newcommand{\quiz}[2]{\noindent\textbf{Name: }\makebox[8cm]{\hrulefill} \hfill \textbf{Algebra II} \\  \textbf{Date: } \hfill \textbf{Quiz #2}\\}

\newcommand{\prob}{\noindent\textbf{Problem. }}
\newcounter{problem}
\newcommand{\problem}{
	\stepcounter{problem}%
	\noindent \textbf{ Problem \theproblem. }        
}
\newcommand{\pspace}{\par\vspace{\baselineskip}}
\newcommand{\ds}{\displaystyle}
\newcommand{\excred}{\noindent\textbf{Extra Credit. }}

% -------------------
% Header & Footer
% -------------------
\usepackage{fancyhdr}

\fancypagestyle{pages}{
	%Headers
	\fancyhead[L]{}
	\fancyhead[C]{}
	\fancyhead[R]{}
\renewcommand{\headrulewidth}{0pt}
	%Footers
	\fancyfoot[L]{}
	\fancyfoot[C]{}
	\fancyfoot[R]{}
\renewcommand{\footrulewidth}{0.0pt}
}
\headheight=0pt
\footskip=14pt

\pagestyle{pages}


% -------------------
% Content
% -------------------
\begin{document}
\quiz{\#}{5}


% Question
%\prob This is an unnumbered problem. \pspace


% Question 1
\problem Let $f(x) = 2x^2 - 4x + 1$. Write $f(x)$ in the form $f(x) = a(x \pm h)^2 \pm k$ by completing the square. Describe how $x^2$ is transformed to obtain $f(x)$. Find the zeros of $f(x)$. Graph $f(x)$, labeling the vertex and all axis intersections.

\vspace{9.5cm}


% Question 2
\problem Let $f(x) = 3x^2 +12x - 63$. Write $f(x)$ in the form $f(x) = a(x \pm h)^2 \pm k$ by completing the square. Describe how $x^2$ is transformed to obtain $f(x)$. Find the zeros of $f(x)$. Graph $f(x)$, labeling the vertex and all axis intersections.

\newpage


% Question 3
\problem Let $f(x) = -5x^2 + 2x - \frac{6}{5}$. Write $f(x)$ in the form $f(x) = a(x \pm h)^2 \pm k$ by completing the square. Describe how $x^2$ is transformed to obtain $f(x)$. Find the zeros of $f(x)$. Graph $f(x)$, labeling the vertex and all axis intersections.

\vspace{10cm}


% Question 4
\problem Let $f(x) = -\frac{1}{2}x^2 - 3x + \frac{1}{2}$. Write $f(x)$ in the form $f(x) = a(x \pm h)^2 \pm k$ by completing the square. Describe how $x^2$ is transformed to obtain $f(x)$. Find the zeros of $f(x)$. Graph $f(x)$, labeling the vertex and all axis intersections.

\newpage


% Question 5
\problem Let $f(x) = x^2 - 6x - 7$. Write $f(x)$ in the form $f(x) = a(x \pm h)^2 \pm k$ by completing the square. Describe how $x^2$ is transformed to obtain $f(x)$. Find the zeros of $f(x)$. Graph $f(x)$, labeling the vertex and all axis intersections.

\vspace{10cm}


% Question 6
\problem Shown below is the graph of the function $f(x)$. Graph the function $g(x) = -3f\left(x-\frac{1}{2}\right) + 1$. Be sure to label the final locations of all 6 of the given points.

\begin{center}
\includegraphics[scale=2.5]{fig1.png}
\end{center}

\vspace{.5cm}


% Extra Credit
\excred Choose the highest and lowest points from your graph of $g(x)$ in Problem 6. Check your work by plugging in the $x$-value of each of those two points into the given formula for $g(x)$, using the given graph of $f(x)$ to obtain the output value of $f(x)$, and verify that the final output is indeed the $y$-value of your chosen point. 

\end{document}