\documentclass[11pt,letterpaper]{article}
\usepackage[lmargin=1in,rmargin=1in,tmargin=1in,bmargin=1in]{geometry}
%For systems of equations
\usepackage{systeme}
%For figures
\usepackage{graphicx}
\usepackage{subcaption}
% -------------------
% Packages
% -------------------
\usepackage{
	amsmath,			% Math Environments
	amssymb,			% Extended Symbols
	enumerate,		    % Enumerate Environments
	graphicx,			% Include Images
	lastpage,			% Reference Lastpage
	multicol,			% Use Multi-columns
	multirow			% Use Multi-rows
}


% -------------------
% Font
% -------------------
\usepackage[T1]{fontenc}
\usepackage{charter}


% -------------------
% Commands
% -------------------
\newcommand{\quiz}[2]{\noindent\textbf{Name: }\makebox[8cm]{\hrulefill} \hfill \textbf{Algebra II} \\  \textbf{Date: } \hfill \textbf{Quiz #2}\\}

\newcommand{\prob}{\noindent\textbf{Problem. }}
\newcounter{problem}
\newcommand{\problem}{
	\stepcounter{problem}%
	\noindent \textbf{ Problem \theproblem. }        
}
\newcommand{\pspace}{\par\vspace{\baselineskip}}
\newcommand{\ds}{\displaystyle}
\newcommand{\excred}{\noindent\textbf{Extra Credit. }}

% -------------------
% Header & Footer
% -------------------
\usepackage{fancyhdr}

\fancypagestyle{pages}{
	%Headers
	\fancyhead[L]{}
	\fancyhead[C]{}
	\fancyhead[R]{}
\renewcommand{\headrulewidth}{0pt}
	%Footers
	\fancyfoot[L]{}
	\fancyfoot[C]{}
	\fancyfoot[R]{}
\renewcommand{\footrulewidth}{0.0pt}
}
\headheight=0pt
\footskip=14pt

\pagestyle{pages}


% -------------------
% Content
% -------------------
\begin{document}
\quiz{\#}{4}


% Question
%\prob This is an unnumbered problem. \pspace


% Question 1
\problem Simplify and factor completely. $$\sqrt{x^5 - x^4} - \sqrt{16x - 16}$$
$$=\sqrt{x^4(x - 1)} - \sqrt{16(x - 1)}$$
$$=x^2\sqrt{x - 1} - 4\sqrt{x - 1}$$
$$=(x^2 - 4)\sqrt{x - 1}$$
$$=(x+2)(x-2)\sqrt{x - 1}$$

\vspace{.5cm}


% Question 2
\problem Simplify. $$\sqrt[3]{y^4} \sqrt[3]{16y^5}$$
$$=\sqrt[3]{y^416y^5}$$
$$=\sqrt[3]{16y^4y^5}$$
$$=\sqrt[3]{16y^9}$$
$$=\sqrt[3]{2^4(y^3)^3}$$
$$=\sqrt[3]{2*2^3(y^3)^3}$$
$$=2y^3\sqrt[3]{2}$$

\vspace{.5cm}


% Question 3
\problem First write as one single radical. Then simplify if possible. $$(x^3)^{\frac{1}{2}} (xy^2)^{\frac{1}{3}} (x^2y^3)^{\frac{1}{6}}$$
$$=(x^3)^{\frac{1}{2}*\frac{3}{3}} (xy^2)^{\frac{1}{3}*\frac{2}{2}} (x^2y^3)^{\frac{1}{6}}$$
$$=(x^3)^{\frac{3}{6}} (xy^2)^{\frac{2}{6}} (x^2y^3)^{\frac{1}{6}}$$
$$=\sqrt[6]{(x^3)^{3}} \sqrt[6]{(xy^2)^{2}} \sqrt[6]{(x^2y^3)^{1}}$$
$$=\sqrt[6]{x^9} \sqrt[6]{x^2y^4} \sqrt[6]{x^2y^3}$$
$$=\sqrt[6]{x^9x^2y^4x^2y^3}$$
$$=\sqrt[6]{x^{13}y^7} \: (\text{this is written as one radical})$$
$$=\sqrt[6]{x(x^2)^6yy^6}$$
$$=x^2y\sqrt[6]{xy}$$
Note that I did not say what flavor of radical to use. You could also have chosen to use, say, a $12^{\text{th}}$ root, giving you
$$=(x^3)^{\frac{1}{2}*\frac{6}{6}} (xy^2)^{\frac{1}{3}*\frac{4}{4}} (x^2y^3)^{\frac{1}{6}*\frac{2}{2}}$$
$$=(x^3)^{\frac{6}{12}} (xy^2)^{\frac{4}{12}} (x^2y^3)^{\frac{2}{12}}$$
$$=\sqrt[12]{(x^3)^{6}} \sqrt[12]{(xy^2)^{4}} \sqrt[12]{(x^2y^3)^{2}}$$
$$=\sqrt[12]{x^{18}} \sqrt[12]{x^4y^8} \sqrt[12]{x^4y^6}$$
$$=\sqrt[12]{x^{18}x^4y^8x^4y^6}$$
$$=\sqrt[12]{x^{26}y^{14}}$$
$$=\sqrt[12]{x^{12}x^{12}x^2y^{12}y^2}$$
$$=x^2y\sqrt[12]{x^2y^2}$$
which is an acceptable answer. But notice that we can further reduce this since
$$x^2y\sqrt[12]{x^2y^2} = x^2y(x^2y^2)^{\frac{1}{12}} = x^2y((xy)^2)^{\frac{1}{12}} = x^2y(xy)^{\frac{2}{12}} = x^2y(xy)^{\frac{1}{6}}=x^2y\sqrt[6]{xy}$$
which is the first answer above.
\vspace{.5cm}


% Question 4
\problem Compute. For full credit write your answer in reduced form. $$\left(\frac{1}{2} + i \frac{\sqrt{3}}{2}\right)^2$$
$$=\left(\frac{1}{2} + i \frac{\sqrt{3}}{2}\right)\left(\frac{1}{2} + i \frac{\sqrt{3}}{2}\right)$$
$$=\frac{1}{4} + i \frac{\sqrt{3}}{4} + i \frac{\sqrt{3}}{4} + i^2 \frac{3}{4}$$
$$=\frac{1}{4} + i \left(\frac{\sqrt{3}}{4} + \frac{\sqrt{3}}{4}\right) - \frac{3}{4}$$
$$=\frac{1}{4} - \frac{3}{4} + i \left(\frac{\sqrt{3}}{4} + \frac{\sqrt{3}}{4}\right)$$
$$= - \frac{2}{4} + i \left(\frac{\sqrt{3}}{4} + \frac{\sqrt{3}}{4}\right)$$
$$= - \frac{1}{2} + i \left(\frac{\sqrt{3}}{4} + \frac{\sqrt{3}}{4}\right)$$
$$= - \frac{1}{2} + i \left( \frac{\sqrt{3} + \sqrt{3}}{4} \right)$$
$$= - \frac{1}{2} + i \left( \frac{2\sqrt{3}}{4} \right)$$
$$= - \frac{1}{2} + i \frac{\sqrt{3}}{2} $$

\vspace{.5cm}


% Question 5
\problem Write in the form $a + ib$. Be sure to show your work! $$\frac{1}{2-3i}$$
$$=\frac{1}{(2-3i)}\frac{(2+3i)}{(2+3i)}$$
$$=\frac{(2+3i)}{(2-3i)(2+3i)}$$
$$=\frac{(2+3i)}{4 + 6i - 6i - 9i^2}$$
$$=\frac{(2+3i)}{4 - 9(-1)}$$
$$=\frac{(2+3i)}{4 + 9}$$
$$=\frac{(2+3i)}{13}$$
$$ = \frac{2}{13} + \frac{3}{13} i$$

\vspace{.5cm}


% Question 6
\problem Write in the form $a + ib$. Be sure to show your work! $$\frac{\sqrt{2} - 3i}{2-i\sqrt{3}}$$
$$= \frac{(\sqrt{2} - 3i)}{(2-i\sqrt{3})} \frac{(2+i\sqrt{3})}{(2+i\sqrt{3})}$$
$$= \frac{(\sqrt{2} - 3i)(2+i\sqrt{3})}{4 + 2i\sqrt{3} - 2i\sqrt{3} - 3i^2} $$
$$= \frac{(\sqrt{2} - 3i)(2+i\sqrt{3})}{4 - 3(-1)} $$
$$= \frac{(\sqrt{2} - 3i)(2+i\sqrt{3})}{4 + 3} $$
$$= \frac{(\sqrt{2} - 3i)(2+i\sqrt{3})}{7} $$
$$= \frac{2\sqrt{2} + i \sqrt{2}\sqrt{3} - 6i - 3i^2\sqrt{3}}{7} $$
$$= \frac{2\sqrt{2} + i \sqrt{6} - 6i - 3(-1)\sqrt{3}}{7} $$
$$= \frac{2\sqrt{2} + i \sqrt{6} - 6i + 3\sqrt{3}}{7} $$
$$= \frac{2\sqrt{2} + 3\sqrt{3} + i \sqrt{6} - 6i}{7} $$
$$= \frac{(2\sqrt{2} + 3\sqrt{3}) + i (\sqrt{6} - 6)}{7} $$
$$=\left(\frac{2\sqrt{2} + 3\sqrt{3}}{7}\right) + i \left(\frac{\sqrt{6} - 6}{7}\right)$$

\vspace{.5cm}


% Question 6
\problem Write in the form $a + ib$. Be sure to show your work! $$\frac{x+iy}{u+iv}$$
$$=\frac{(x+iy)}{(u+iv)}\frac{(u-iv)}{(u-iv)}$$
$$= \frac{xu - ixv + iyu - i^2yv}{u^2 - iuv + iuv - i^2v^2}$$
$$= \frac{xu - ixv + iyu - (-1)yv}{u^2 - (-1)v^2}$$
$$= \frac{xu - ixv + iyu +yv}{u^2 +v^2}$$
$$= \frac{xu  +yv+ iyu - ixv }{u^2 +v^2}$$
$$= \frac{(xu  +yv)+ i(yu - xv) }{u^2 +v^2}$$
$$=\left( \frac{xu  +yv}{u^2 + v^2}\right) + i \left(\frac{yu - xv}{u^2 + v^2} \right)$$

\vspace{.5cm}


% Extra Credit
\excred Compute $$\sqrt{2 + \sqrt{2 + \sqrt{2 + \dots}}}$$
Let $x = \sqrt{2 + \sqrt{2 + \sqrt{2 + \dots}}}$ and notice that we must have $0<x$ since at every level, this is the square root of a positive number and therefore positive. Then
$$x = \sqrt{2 + \sqrt{2 + \sqrt{2 + \dots}}}$$
$$ \implies x^2 = 2 + \sqrt{2 + \sqrt{2 + \sqrt{2 + \dots}}}.$$
But we can substitute $x = \sqrt{2 + \sqrt{2 + \sqrt{2 + \dots}}}$ to give us
$$ x^2 = 2 + \left(\sqrt{2 + \sqrt{2 + \sqrt{2 + \dots}}}\right)$$
$$x^2 = 2 + x$$
$$\iff x^2 - x - 2 = 0$$
$$\iff (x+1)(x-2) = 0$$
which is true whenever $x = -1$ or when $x = 2$. But since we must have $0<x$, it must be the case that $x = 2$.  We'll check our work by letting $x=2$ in $x = \sqrt{2 + \sqrt{2 + \sqrt{2 + \dots}}}$ so that
$$ 2 =  \sqrt{2 + \sqrt{2 + \sqrt{2 + \dots}}}$$
$$\implies 2^2 = 2 + \sqrt{2 + \sqrt{2 + \sqrt{2 + \dots}}}$$
$$\iff 4 = 2 + \sqrt{2 + \sqrt{2 + \sqrt{2 + \dots}}}$$
$$\iff 4 - 2 = 2 - 2 + \sqrt{2 + \sqrt{2 + \sqrt{2 + \dots}}}$$
$$\iff 2 = 0 + \sqrt{2 + \sqrt{2 + \sqrt{2 + \dots}}}$$
$$\iff 2 = \sqrt{2 + \sqrt{2 + \sqrt{2 + \dots}}}$$
which is what we want. Therefore, 
$$\sqrt{2 + \sqrt{2 + \sqrt{2 + \dots}}} = 2.$$

\end{document}